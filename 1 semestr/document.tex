\documentclass[12pt]{article} 
\usepackage[left=20mm, top=16mm, right=16mm, bottom=20mm]{geometry} 
\usepackage{graphicx}
\usepackage{wrapfig}
\graphicspath{{pictures/}}
\DeclareGraphicsExtensions{.pdf,.png,.jpg,.eps}
\usepackage{cmap}					% поиск в PDF
\usepackage{mathtext} 				% русские буквы в формулах
\usepackage[T2A]{fontenc}	
\usepackage[utf8x]{inputenc} 
\usepackage[russian]{babel} 
\usepackage{amsmath,amsfonts,amssymb,amsthm,mathtools} 
\usepackage{icomma} % "Умная" запятая: $0,2$ --- число, $0, 2$ --- перечисление
\usepackage{euscript}	 % Шрифт Евклид
\usepackage{mathrsfs} % Красивый матшрифт

%% Свои командыs
%\DeclareMathOperator{\sgn}{\mathop{sgn}}
\newcommand{\te}{\ensuremath{\Rightarrow}}
\newcommand{\y}{\ensuremath{\angle}}
\newcommand{\ABC}{\ensuremath{\triangle ABC\,}}
\newcommand{\tr}{\ensuremath{\triangle}}
\newcommand{\ca}{\ensuremath{\cos\alpha}}
\newcommand{\sa}{\ensuremath{\sin\alpha}}
\newcommand{\cb}{\ensuremath{\cos\beta}}
\newcommand{\sib}{\ensuremath{\sin\beta}}
\newcommand{\x}{\cdot}
%% Перенос знаков в формулах (по Львовскому)
\newcommand*{\hm}[1]{#1\nobreak\discretionary{}
	{\hbox{$\mathsurround=0pt #1$}}{}}

\begin{document}

\begin{titlepage}		
\begin{center}
\large 	Московский физико-технический университет \\
Факультет общей и прикладной физики \\
\vspace{4.5cm}
\large Вопрос по выбору в 1 семестре \\(Общая физика: механика) \\ \vspace{0.7cm}
\LARGE \textbf{Смещение перигелия Меркурия}
\end{center}
\vspace{2.3cm} \large

\begin{center}
		 Автор: \\
 Иванов Кирилл,
 625 группа
\vspace{10mm}

Семинарист:

Слободянин Валерий Павлович
\end{center}

\begin{center} \vspace{70mm}
	г. Долгопрудный, 2016 год
\end{center}
\end{titlepage}

\section{Введение}
\textbf{Аномальное смещение перигелия Меркурия} — обнаруженная в 1859 году французским физиком и астрономом Урбеном Леверье особенность движения планеты Меркурий, сыгравшая исключительную роль в истории физики. 
Это смещение оказалось первым движением небесного тела, которое не подчинялось ньютоновскому закону всемирного тяготения.

Дело в том, что Леверье установил, что в рамках классической теории гравитации Ньютона прецессия перигелия должна составлять $ 526,7'' $ за столетие (что мы покажем в пункте 4), в то время как наблюдаемое смещение составляло почти $ 565'' (570''$  по современным наблюдениям). Полученная разница в почти $ 40'' $ за столетие хотя и невелика, но значительно превышает погрешности наблюдения и нуждается в объяснении.

Таким образом, физики были поставлены перед необходимостью искать пути модифицировать или обобщить теорию тяготения.
Поиски увенчались успехом в 1915 году, когда Альберт Эйнштейн разработал общую теорию относительности (ОТО); из уравнений ОТО вытекало именно такое значение смещения, которое фактически наблюдалось. Позже были измерены аналогичные смещения орбит нескольких других небесных тел, значения которых также совпали с предсказанными ОТО. 

\section{Классическая теория гравитации}
Для начала вспомним формулировку главного закона Ньютоновской гравитации: 
\begin{equation}
\mathbf{F} = -G \dfrac{mM}{r^3}\mathbf{r}
\end{equation}
Из него следует, что потенциальная энергия взаимодействия двух тел в центральном поле гравитации $ \Pi = - G\frac{mM}{r}$, и с помощью закона сохранения энергии получаем: 
\begin{equation}
\varepsilon = K + \Pi = \frac{v^2}{2} - G\frac{M}{r} = const
\end{equation}
где $ \varepsilon = \frac{E}{m} $ --- удельная энергия тела. Вспомним и о законе сохранения момента импульса (его мы также возьмем удельным): 
\begin{equation}
\lambda = vr = const
\end{equation}

Вместе (2) и (3) позволяют решить задачу двух тел в классическом случае. 

\vspace{0.1cm}
В курсе механики в МФТИ при выводе законов движения тел в гравитационном поле опираются на законы Кеплера, сразу постулируя, что траектория движения является кривой второго порядка (эллипсом, гиперболой или параболой). Мы же выведем это из общих соображений, опираясь только на законы сохранения, написанные выше.

\section{Вывод закона движения планеты}
Введем полярную систему координат $ (r,\varphi) $ с центром в положении Солнца $ (S) $. Пусть планета движется по произвольной траектории со скоростью $ v $, разложим ее на 2 составляющие: радиальную $ v_r = \dot{r} $ и азимутальную $ v_\varphi = r\dot{\varphi}  $, и тогда $ v^2 = \dot{r}^2 + r^2\dot{\varphi}^2 $. Закон сохранения энергии принимает вид:
\begin{equation}
\varepsilon = \frac{1}{2}(\dot{r}^2 + r^2\dot{\varphi}^2) - G\frac{M}{r} = const
\end{equation}
  \begin{wrapfigure}{l}{0.4\linewidth} 
  	\includegraphics{1}
  	\caption{Траектория планеты}
  \end{wrapfigure}
Пусть радиус вектор заметает площадь $ dS $ за время $ dt $, тогда \\ $ dS = \frac{1}{2}{r_1}, r_2] =\frac{1}{2}[r_1, r_1 + vdt]  = \frac{1}{2}[r_1, v_\varphi]dt \te $ 
\begin{equation}
\dfrac{dS}{dt} = \sigma = \dfrac{r^2\dot{\varphi}}{2} = \dfrac{\lambda}{2}  =const
\end{equation}

Чтобы найти траекторию движения планет, нам нужно исключить время из уравнений ЗСЭ (4) и ЗСМИ (5).

Считая $ r = r(\varphi), \te \dot{r} = \frac{dr}{d\varphi} \dot{\varphi}$. Подставим это в (4) и используя, что $ \dot{\varphi} = \frac{2\sigma}{r^2} $ из (5), мы получаем: 
\begin{equation}
\left(\dfrac{1}{r^2}\dfrac{dr}{d\varphi}\right)^2 + \dfrac{1}{r^2} = \dfrac{1}{2\sigma^2}\left(\varepsilon + G\dfrac Mr\right)
\end{equation}

Введем переменную $ \rho = -\frac 1r + \frac 1p, p $ --- некая константа.

 Тогда перепишем (6) как:
\begin{equation}
\left( \dfrac{d\rho}{d\varphi}\right)^2 + \left(\rho - \dfrac 1p \right)^2 = \dfrac{\varepsilon}{2\sigma^2} + G \dfrac{M}{2\sigma^2} \left(-\rho +\frac 1p  \right)
\end{equation}

Возьмем $ p $ такой, чтобы уничтожить члены, содержащие первую степень переменной $ \rho $:
\begin{equation}
p = \dfrac{4\sigma^2}{GM}
\end{equation}

Наше уравнение (7) примет следующий вид: 
\begin{equation}
\left( \dfrac{d\rho}{d\varphi}\right)^2 = \dfrac{\varepsilon}{2\sigma^2} + \dfrac{1}{p^2} - \rho^2
\end{equation}
 
 Заметим, что так как правая часть уравнения (9) неотрицательна, то и константа $ \frac{\varepsilon}{2\sigma^2} + \frac{1}{p^2} $ должна быть положительной. Обозначим ее за $ A^2 $:
 \begin{equation}
 \left( \dfrac{d\rho}{d\varphi}\right)^2  = A^2 - \rho^2
 \end{equation}
 
 Из вышесказанного очевидно, что $ A^2 \geq \rho^2 \te \frac{\rho}{A} \leq 1 $. Тогда можно взять это отношение как $\frac{\rho}{A} =  \cos \theta $, где $ \theta $ --- некая новая переменная. Из тригонометрии и продифференцировав $ \frac{\rho}{A} $ по $ \varphi $ получаем: 
 \begin{equation}
A^2 - \rho^2 = A^2 \sin^2 \theta,   \dfrac{d\rho}{d\varphi} = -A\sin\theta \dfrac{d\theta}{d\varphi}
 \end{equation}
 
 Подставив (11) в (10) и сокращая на $ A^2\sin^2\theta $ ,мы получаем $ (\frac{d\theta}{d\varphi})^2 = 1 $, откуда 
 \begin{equation}
 \theta  = \int\limits_{\varphi_0}^{\varphi}\pm d\varphi= \pm \varphi + \varphi_0
 \end{equation}
 
 Тогда из определения получаем, что $ \rho = A\cos(\pm\varphi + \varphi_0) = A\cos(\varphi \pm \varphi_0) $ из свойства четности косинуса. Так как $ \varphi_0 $ это константа интегрирования, которая может принимать любой знак (что мы уже использовали в (12)), то нам нет смысла сохранять чередование знаков в полученном выражении для $ \rho $. Таким образом, 
 \begin{equation}
 \rho  = A\cos(\varphi + \varphi_0) 
 \end{equation}
 
 Введем $ e = pA $ и перепишем определение $ \rho $: 
 \begin{equation}
 \rho = \dfrac 1p - \dfrac 1r \te \dfrac 1r = \dfrac 1p \left( 1 - \dfrac eA \rho\right) 
 \end{equation}
 
 Используя результат из (13), мы получим окончательный ответ для траектории движения планеты в общем виде: 
 \begin{equation}
 \fbox{ $ r = \dfrac{p}{ 1 - e\cos(\varphi + \varphi_0)  }
	   	 $}
 \end{equation}
 
 Из рис. 1 мы можем взять $ \varphi_0 = 0 $, то есть координата $ \varphi $ откладывается от оси $ Sx $ в сторону движения против часовой стрелки. Тогда наше уравнение (15) примет вид: 
 \begin{equation}
 \fbox{ $ r = \dfrac{p}{ 1 - e\cos\varphi   }
 	$}
 \end{equation}
 
 Рассмотрим внимательнее, что значит константа $ e $. 
 \begin{equation}
 e = pA = \sqrt{1 + \dfrac{\varepsilon p^2}{2\sigma^2}} = \sqrt{1+ \dfrac{8\varepsilon\sigma^2}{G^2M^2}}
 \end{equation}
 
 Таким образом, опираясь только на законы сохранения и классическую теорию гравитации Ньютона(не используя законов Кеплера вовсе), \textbf{мы получили уравнение конического сечения в полярной системе координат}. Тогда константа $ e $ --- эксцентриситет этой кривой, и в зависимости от его значения траектория движения нашей планеты будет \textbf{эллипсом} ($ e < 1, \varepsilon < 0 $), \textbf{параболой} ($ e = 1, \varepsilon = 0 $) или \textbf{гиперболой} ($ e >1, \varepsilon >0 $).
 
 Из наблюдений нам известно, что траектория движения планет Солнечной системы --- это \textbf{эллипс} (так как движение планет финитно). Однако, как упоминалось в пункте 1.1, те же наблюдения показали, что это не совсем верно, и параметры орбит планет Солнечной системы из-за взаимовлияния этих планет со временем претерпевают медленные изменения. 
 
 Для Меркурия, в частности, ось его орбиты постепенно поворачивается (в плоскости орбиты) в сторону орбитального движения, соответственно, смещается и ближайшая к Солнцу точка орбиты — перигелий («прецессия перигелия»). Рассмотрим внимательнее этот процесс.  
 
 \section{Прецессия перигелия Меркурия}

 \begin{wrapfigure}{l}{0.48\linewidth} 
  	\includegraphics{3}
  	\caption{Взаимодействие планет и наклонная орбита}
  \end{wrapfigure}
 
 Рассмотрим движение Меркурия. Очевидно, на него действует со стороны Солнца сила $ \textbf{S} \hm{=} - G\frac{mM}{r^3}\textbf{r} $. Наблюдения показывают, что мы не можем полностью пренебрегать гравитационным влиянием других планет. Очевидно, что на Меркурий наибольшее влияние оказывает ближайший сосед --- Венера. Пусть сила их взаимодействия $ \textbf{V} $, тогда формула (1) примет вид: 
 \begin{equation}
\textbf{ F} = -G\dfrac{mM}{r^3}\textbf{r} + \textbf{V}
 \end{equation}
 
 Строго говоря, на Марс действуют и другие планеты. Так, как мы увидим ниже, наибольший вклад вносят Земля и Юпитер, то есть более правильно переписать уравнение (23) в следующем виде: 
 \begin{equation}
 \textbf{ F} = -G\dfrac{mM}{r^3}\textbf{r} + \textbf{V} + \textbf{E} + \textbf{U} + ...
 \end{equation}

  \begin{wrapfigure}{l}{0.38\linewidth} 
  	\includegraphics{4}
  	\caption{Прецессия эллиптической орбиты}
  \end{wrapfigure}
 
 Как мы показали в пункте 3, опираясь на (1) и законы сохранения (2), (3), можно получить их точное и строгое решение в виде уравнений (15) и (16), которые задают траекторию тела как кривую второго порядка (эллипс для планет Солнечной системы).
 
 Однако уже в формуле (18), не говоря о более сложном выражении (19), мы получаем задачу о гравитационном взаимодействии 3 и более тел, которая не имеет такого решения.
 Из математики известно, что в теории возмущений искомое решение находится через сумму рядов, что позволяет получить приближенное решение, в котором точность зависит от количества членов разложения. 
 
 Эта задача была довольна сложна в XIX веке, ведь тогда не существовало ЭВМ и компьютерных технологий, которые сейчас позволяют решать даже очень большие и громоздкие выражения. Однако с помощью сохранения большого числа членов ряда уже тогда ученые смогли получить, что решением задачи многих тел является незамкнутый, <<прецессирующий>> эллипс, который изображен на рис. 3. 
 
 При этом угол между двумя последующими положениями перигелия равен не  $360^\circ $, а $ 360^\circ + d\varphi$, и говорят, что перигелий за каждый орбитальный оборот планеты смещается на $ d\varphi $ --- орбита прецессирует.
 
 Спустя много лет наблюдений за движением планет Леверье в 1859 году смог получить полные величины планетарных возмущений, учитывая влияние разных планет. Конкретно, распишем вклад в смещение перигелия Меркурия каждой планетой: 
 
 \begin{center}
 	\begin{tabular}[t]{ |c|c|c| }
\hline
  \textnumero   & Планета & Вклад в смещение \\
\hline
1& Венера & $ 280,6''  $ \\
2& Земля& $ 83,6'' $ \\
3& Юпитер & $ 152,6'' $ \\
4& Марс & $ 2,6'' $ \\
5& Сатурн & $7,2''$ \\
6& Уран & $0,1''$ \\
\hline
\multicolumn{3}{|c|}{Итоговое смещение за 100 лет $\varDelta\varphi= 526,7'' $}\\
\hline
\end{tabular}
 \end{center}

Французским ученым была проделана огромная работа. К сожалению, разница с наблюдаемым результатом смещения $ 565'' $ заставила физиков более чем 50 лет искать иной способ вычисления смещения перигелия Меркурия. В конечном итоге это решение было найдено с помошью общей теории относительности (ОТО) Альберта Эйнштейна.

\section{Решение в рамках общей теории относительности}
После создания в 1905 году специальной теории относительности (СТО) А. Эйнштейн осознал необходимость разработки релятивистского варианта теории тяготения, поскольку уравнения Ньютона были несовместимы с преобразованиями Лоренца, а скорость распространения ньютоновской гравитации была бесконечна. 

Первые наброски релятивистской теории тяготения опубликовали в начале 1910-х годов Макс Абрахам, Гуннар Нордстрём и сам Эйнштейн. У Абрахама смещение перигелия Меркурия было втрое меньше реального, в теории Нордстрёма ошибочным было даже направление смещения, версия Эйнштейна 1912 года давала значение на треть меньше наблюдаемого.

В 1915 году Эйнштейн опубликовал окончательный вариант своей новой теории тяготения, получившей название «общая теория относительности» (ОТО). В ней, в отличие от ньютоновской модели, вблизи массивных тел геометрия пространства-времени заметно отличается от евклидовой, что приводит к отклонениям от классической траектории движения планет.

18 ноября 1915 года Эйнштейн рассчитал (приближённо) это отклонение и получил практически точное совпадение с наблюдаемыми 43″ в столетие. При этом не понадобилось никакой подгонки констант и не делалось никаких произвольных допущений. Приведем без вывода эту формулу: 
\begin{equation}
\delta\varphi \approx \dfrac{6\pi GM}{c^2a(1-e^2)} = 42,98''
\end{equation}

Здесь $ M $ --- масса Солнца, $ a $ --- большая полуось орбиты Меркурия, $ e $ -- ее эксцентриситет,\\ $ c $ --- скорость света в вакууме,  а $ \delta\varphi $ --- это дополнительная поправка к смещению, которое вычислил Леверье.

Точный расчет смещения перигелия Меркурия послужил первым признаком справедливости ОТО. Впоследствие по этой формуле были получены соответствующие наблюдениям смещения для других планет, а ОТО была подтверждена и другими экспериментами и наблюдениями.  

\section{Вывод}

Таким образом, на этом примере мы убедились, что в теории физика может построить сколь угодно строгую и красивую модель. Однако, если эта самая модель не соотвествует эксперименту, то ее необходимо заменить на другую, ту, которая будет находится в согласии с опытом. При этом физики обязаны поступить именно так, даже если новая модель кажется им более странной, противоречивой и нелогичной.

\section*{Список литературы} \large
\begin{enumerate}
  \item Роузвер Н. Т., Перигелий Меркурия. От Леверье до Эйнштейна = Mercury's perihelion. From Le Verrier to Einstein, М.: Мир (1985).
  \item Сивухин Д. В., Общий курс физики: Учебное пособие для вузов, В 5 т. Т. 1. Механика. --- 5-е изд., стереот., М.: ФИЗМАТЛИТ, (2006).
  \item Субботин М. Ф., Введение в теоретическую астрономию. — М.: Наука (1968).
  \end{enumerate}



\end{document}

