\documentclass[12pt]{kiarticle} 
\graphicspath{{../pictures/}}
\DeclareGraphicsExtensions{.pdf,.png,.jpg,.eps}
\usepackage{indentfirst}
\newcommand{\del}{\ensuremath{\delta}}
\newcommand{\co}{\ensuremath{\mathrm{const}}}
%%%
\fancyhead[L]{Вопрос по выбору --- термодинамика, 2017\hfil}
\fancyhead[R]{\hfil Иванов Кирилл, 625 группа }



\begin{document}

\begin{titlepage}
	\begin{center}
		\large 	Московский физико-технический университет \\
		Факультет общей и прикладной физики \\
		\vspace{0.2cm}
		
		\vspace{4.5cm}
		Вопрос по выбору во 2 семестре \\ \vspace{0.2cm}
		\large (Общая физика: термодинамика) \\ \vspace{0.2cm}
		\LARGE \textbf{Пока хз еще}
	\end{center}
	\vspace{2.3cm} \large
	
	\begin{center}
		Автор: \\
		Иванов Кирилл,
		625 группа
		\vspace{10mm}
		
		Семинарист: 
		
		Слободянин Валерий Павлович
		
		
	\end{center}
	
	\begin{center} \vspace{50mm}
		г. Долгопрудный \\
		 2017 год
	\end{center}
\end{titlepage}

%%%%%%%%%%
%%%%%%%%%%
%%%%%%%%%%%%%%%%%%%%%%%%%%%%%%%%%%%

\section{Введение}

В курсе термодинамики, изучаемом в МФТИ, постулируют сначала так называемое "<нулевое"> и затем первое начала термодинамики, а затем, изучая тепловые машины и циклы, приходят ко второму началу термодинамики. В процессе такого перехода весьма естественно формируется определение \textbf{энтропии} как приведённого тепла: 

\begin{equation}\label{defent}
 dS = \dfrac{\del Q}{T}
\end{equation}

Впоследствии с помощью этого получается обобщение первого начала термодинамики в виде $ TdS = dU + \del A $ и энтропии как одной изи четырех термодинамических потенциалов. 

Мы же попробуем ввести понятие энтропии немного по-другому, опираясь на другие понятия.

\section{Вывод энтропии}

\subsection{Определение температуры}

Рассмотрим произвольный сосуд с некоторым газом. Весьма естественно определить такие макропараметры, как \textbf{давление} $ P $ и \textbf{объем} $ V $ из чисто логических и механических соображений. Введём также понятие \textbf{условной температуры} $ \tau $ как некого третьего параметра нашей системы и будем считать его "<мерой нагретости тела"> (понимая, однако, наличие и более объективного смысла температуры, на котором мы не будем останавливаться в данной работе). 

Зафиксировав этот параметр, будем изменять $ P $ и $ V $, и так сделаем для разных температур. Опыты показывают, что для каждой температуры можно построить кривую (\textbf{изотерму}) на плоскости $ PV $, которые не будут пересекаться, и при нормальных условиях будут приближённо иметь вид гипербол. 

Тогда такой газ мы назовём \textbf{идеальным}, а условная температура $ \tau = \tau (P, V) $ будет являться функцией состояния системы. Кроме того, для уравнения состояния будет справедливо следующее:

\begin{equation}\label{}
PV = \co \te \tau (P, V) = \tau (PV)
\end{equation}

Так можно сформулировать \textit{принцип температуры} о существовании функции состояния, остающейся неизменной при любом процессе на изотерме.











%\section*{Список литературы}
%
%\begin{enumerate}
%	
%	\item 
%	
%\end{enumerate}














\end{document}
	
