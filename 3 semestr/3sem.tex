\documentclass[12pt]{kiarticle}
\graphicspath{{pictures/}}
\DeclareGraphicsExtensions{.pdf,.png,.jpg,.eps}
%%%
\pagestyle{fancy}
\fancyhf{}
%\renewcommand{\headrulewidth}{ 0.1mm }
\renewcommand{\footrulewidth}{ .0em }
\fancyfoot[C]{\texttt{\textemdash~\thepage~\textemdash}}
\fancyhead[L]{Вопрос по выбору --- электричество, 2018\hfil}
\fancyhead[R]{\hfil Иванов Кирилл, 625 группа }
\usepackage{multirow} % Слияние строк в таблице
\newcommand
{\un}[1]
{\ensuremath{\text{#1}}}
\newcommand{\eds}{\ensuremath{ \mathscr{E}}}
\usepackage{tikz}
%%% Работа с таблицами
\usepackage{array,tabularx,tabulary,booktabs} % Дополнительная работа с таблицами
\usepackage{longtable}  % Длинные таблицы
\usepackage{multirow} % Слияние строк в таблице

\begin{document}
	
	\begin{titlepage}
	\begin{center}
		\large 	Московский физико-технический институт \\
		(государственный университет) \\
		Факультет общей и прикладной физики \\
		\vspace{0.2cm}
		
		\vspace{4.5cm}
		Вопрос по выбору в 3 семестре \\ \vspace{0.2cm}
		\large (Общая физика: электричество и магнетизм) \\ \vspace{0.2cm}
		\LARGE \textbf{Свойства и приложения уравнений Максвелла}
	\end{center}
	\vspace{2.3cm} \large
	
	\begin{center}
		Работу выполнил: \\
		Иванов Кирилл,
		625 группа
		\vspace{10mm}		
		
	\end{center}
	
	\begin{center} \vspace{60mm}
		г. Долгопрудный \\
		2018 год
	\end{center}
\end{titlepage}



\section{Введение}

Запишем уравнения Максвелла в дифференциальной и интегральной формах: 


\begin{equation}\label{Gauss_E}
\di \mathbf{D} = 4\pi\rho, \qquad
\oint\limits_S  \mathbf{D} \x d\mathbf{S} = 4\pi q
\end{equation}
\begin{equation}\label{Gauss_B}
\di \textbf{B} = 0 , \qquad
\oint\limits_S  \mathbf{B} \x d\mathbf{S} =  0
\end{equation}
\begin{equation}\label{Faradey}
\rot \mathbf{E} = -\dfrac{1}{c}\pdd{\mathbf{B}}{t}, \qquad \oint\limits_\Gamma \mathbf{E} \x d\mathbf{l} = - \dfrac{1}{c} \int\limits_S  \pdd{\mathbf{B}}{t} \x d\mathbf{S}
\end{equation}
\begin{equation}\label{Circ_H}
\rot \mathbf{H} = \dfrac{4\pi}{c}\mathbf{j} + \dfrac{1}{c}\pdd{\mathbf{D}}{t}, \qquad
\oint\limits_\Gamma \mathbf{H} \x d\mathbf{l} = \dfrac{4\pi}{c}\mathbf{I} + \dfrac{1}{c} \int\limits_S  \pdd{\mathbf{D}}{t} \x d\mathbf{S} 
\end{equation}

В курсе общей физики МФТИ эти уравнения изучаются и рассматриваются в основном именно в такой форме. Попробуем посмотреть на них немного по-другому и получить из этого какие-то полезные результаты. 


\section{Потенциалы электромагнитного поля}

\subsection{Определение потенциалов}

При изучении электростатического поля мы говорили о его потенциальном характере и вводили \textbf{скалярный потенциал электрического поля}:

\[ 
\int\limits_1^2 \mathbf{E(r)} \x d\mathbf{r} = \phi_1 - \phi_2 \te \phi(\mathbf{r}) - \phi(\textbf{r} + d\mathbf{r}) = -d\phi = \mathbf{E(r)} \x d\mathbf{r}
 \]

Отсюда получаем равенство поля градиенту потенциала (здесь и далее будем использовать векторный оператор "<набла"> $ \nabla $, а также рассматривать только дифференциальный вид уравнений Максвелла с использованием вектора "<набла">):

\begin{equation}\label{phi}
\mathbf{E} = - \nabla \phi 
\end{equation}

Похожий потенциал, только векторный, можно ввести и для магнитного поля. При помощи математических преобразований из закона Био-Савара-Лапласа можно получить (например, как в учебном пособии Н.А. Кириченко) равенство 

\begin{equation}\label{A}
\mathbf{B} = \nabla \times \mathbf{A}
\end{equation}

где $ \mathbf{A} $ --- \textbf{векторный потенциал магнитного поля}, определяемый по формуле

\begin{equation}\label{A_def}
\mathbf{A}(\mathbf{r}) = \dfrac{1}{c} \int\limits_V \dfrac{\mathbf{j}(\mathbf{r_1})}{|\mathbf{r} - \mathbf{r_1}|} dV_1
\end{equation}

\subsection{Потенциал в уравнениях Максвелла} \label{pot_th}

\paragraph*{Докажем теперь следующее:} уравнения Максвелла \eqref{Gauss_B} и \eqref{Faradey} выполняются тождественно, если электрическое и магнитное поля выражены через потенциалы следующим образом:

\begin{equation}\label{pot}
\mathbf{E} = - \nabla \phi  - \dfrac{1}{c} \pdd{\mathbf{A}}{t}, \qquad \mathbf{B} = \nabla \times \mathbf{A}
\end{equation}

Здесь в электрическом поле по сравнению с \eqref{phi} есть добавка, зависящая от переменного по времени магнитного поля (что весьма естественно, учитывая физический смысл закона Фарадея).

\paragraph*{Док-во:} В векторном анализе существует такая математическая теорема: любое дифференцируемое векторное поле может быть разложено на две составляющих, равных градиенту скалярного потенциала и ротору векторного потенциала поля (теорема Гельмгольца). Воспользовавшись ей, мы получаем из \eqref{Gauss_B} равенство $ \mathbf{B} = \nabla \times \mathbf{A} $ (ведь скалярного потенциала у вихревого поля $ \mathbf{B} $ нет). 

Теперь введем $ \mathbf{E}' = \mathbf{E} + \frac{1}{c} \pdd{\mathbf{A}}{t} $, тогда ротор этого вектора равен нулю. В самом деле:

\begin{equation}\label{}
\nabla \times \mathbf{E}' = \nabla \times \mathbf{E} + \dfrac{1}{c}\nabla \times \pdd{\mathbf{A}}{t} = -\dfrac{1}{c} \pdd{\mathbf{B}}{t} +   \dfrac{1}{c}\pdd{}{t} \nabla \times \mathbf{A} = -\dfrac{1}{c} \pdd{\mathbf{B}}{t} + \dfrac{1}{c} \pdd{\mathbf{B}}{t} = 0
\end{equation}

Здесь мы воспользовались \eqref{Faradey} и только что полученным определением $ \mathbf{B} $. Тогда из равенства нулю этого ротора мы по теореме Гельмгольца получаем $ \mathbf{E}' = - \nabla \phi $, откуда и следует искомое определение через потенциалы \eqref{pot}.

В обратную сторону все доказывается аналогично: из \eqref{pot} мы получаем \eqref{Gauss_B} по свойству равенства нулю смешанного произведения с двумя равными векторами ("<набла">), а для  \eqref{Faradey} мы также получаем ноль у смешанного произведения со скалярным потенциалом и производную $ \mathbf{B} $ по времени из его определения через ротор $ \mathbf{A} $.

Таким образом, так как мы считаем верными уравнения Максвелла, то мы всегда можем выразить электрическое и магнитное поля через их потенциалы.

\subsection{Калибровка потенциала}

Заметим теперь, что при определении полей \eqref{pot} наши потенциалы заданы неоднозначно. В самом деле, если мы совершим следующие "<сдвиги"> (преобразования) потенциалов: 

\begin{equation}\label{psi}
\mathbf{A} \st \mathbf{A} + \nabla \psi, \qquad \phi \st \phi - \dfrac{1}{c} \pdd{\psi}{t}
\end{equation}

где некая функция $ \psi = \psi(\mathbf{r}, t) $, то наши поля в \eqref{pot} не изменятся. В самом деле, для $ \mathbf{B}  $ это следует из свойства векторного произведения, а для $ \mathbf{E} $ наши добавки в виде $ \frac{1}{c} \nabla \pdd{\psi}{t} $ сократятся. 

Такие преобразования играют важную роль в теоретической физике (а именно, в квантовой электродинамике они лежат в основе локальной калибровочной симметрии электромагнитного воздействия). Например, это дает возможность с помощью теоремы Нётер получить закон сохранения заряда. 

Весьма удобным способом задать однозначность потенциалов является наложение дополнительных условий (которые называются \textbf{калибровкой потенциалов}). Точнее говоря, удобным является то, что мы можем выбрать их так, чтобы они удовлетворяли произвольному условию, которое мы можем менять в зависимости от наших целей. Рассмотрим, например, лоренцеву калибровку (здесь и в следующем пункте работаем в вакууме, $ \mu = \epsilon = 1, \mathbf{D} = \mathbf{E}, \mathbf{B} = \mathbf{H} $): 

\begin{equation}\label{Lorents}
\nabla \mathbf{A}+ \dfrac{1}{c} \pdd{\phi}{t} = 0
\end{equation}

Понятно, что при преобразовании \eqref{psi} это уравнение уже не будет выполнятся, что нам и требовалось.



\subsection{Волновые уравнения для потенциалов}

Получим теперь оставшиеся уравнения Максвелла \eqref{Gauss_E}, \eqref{Circ_H} через наши потенциалы. Для этого возьмем ротор от уравнения \eqref{Faradey} и раскроем его по правилу двойного векторного произведения:

\begin{equation}\label{}
\nabla \times \nabla \times \mathbf{E} = \nabla (\nabla \x \mathbf{E}) - (\nabla \x \nabla) \mathbf{E} = - \dfrac{1}{c} \pdd{}{t}\nabla \times \mathbf{B}
\end{equation}

Используя обозначения $ \triangle = \nabla \x \nabla $  --- оператор Лапласа, $ \square = \triangle -  \frac{1}{c^2}\pdd{^2}{t^2}$ --- оператор Д'Аламбера, а также подставив вместо первого члена слева уравнение \eqref{Gauss_E}, а вместо последнего члена справа уравнение \eqref{Circ_H}, мы получаем:

\begin{equation}\label{sq_E}
\dfrac{1}{c^2}\pdd{^2\mathbf{E}}{t^2} - \triangle \mathbf{E} = - \dfrac{4\pi}{c^2} \pdd{\mathbf{j}}{t} - 4 \pi \nabla \rho \te \square \mathbf{E} = \dfrac{4\pi}{c^2} \pdd{\mathbf{j}}{t} + 4 \pi \nabla \rho
\end{equation}

Аналогично получим и уравнение для магнитного поля (взяв ротор от \eqref{Circ_H} и подставив туда уравнения \eqref{Faradey} и \eqref{Gauss_B}):

\begin{equation}\label{sq_B}
\dfrac{1}{c^2}\pdd{^2\mathbf{B}}{t^2} - \triangle \mathbf{B} = \dfrac{4\pi}{c^2} \nabla \times \mathbf{j} \te \square \mathbf{B} = - \dfrac{4\pi}{c} \nabla \times \mathbf{j}
\end{equation}

Таким образом, мы всего лишь записали волновные уравнения электромагнитного поля, явно учитывая источники поля --- заряды $ \rho $ и токи $ \mathbf{j} $. Подставим теперь в наши уравнения \eqref{sq_E} -- \eqref{sq_B} потенциалы \eqref{pot} (опираясь на доказанное в пункте \ref{pot_th}). Получаем:

\begin{equation}\label{sq}
\left.
\begin{aligned}
& \square \left( - \nabla \phi  - \dfrac{1}{c} \pdd{\mathbf{A}}{t} \right) = \dfrac{4\pi}{c^2} \pdd{\mathbf{j}}{t} + 4 \pi \nabla \rho \\
& \square \left( \nabla \times \mathbf{A} \right) = \nabla \times - \dfrac{4\pi}{c}\mathbf{j} \\
\end{aligned}
\right.
\end{equation}

Замечаем, что последнее уравнение будет верно тождественно, если положить следующее: 

\begin{equation}\label{sq_A}
\square \mathbf{A} = - \dfrac{4\pi}{c}\mathbf{j}
\end{equation}

Тогда с учётом \eqref{sq_A} мы получаем для первого из \eqref{sq} следующий вид уравнения:

\begin{equation}\label{}
\square (\nabla \phi) = -4\pi \nabla \rho
\end{equation}

Аналогично \eqref{sq_A} запишем:

\begin{equation}\label{sq_phi}
\square \phi = - 4 \pi \rho
\end{equation}

Таким образом, мы получили \eqref{sq_A}, \eqref{sq_phi} --- волновые уравнения для потенциалов электромагнитного поля, которые вместе с определениями полей через потенциалы \eqref{pot} полностью эквивалентны исходным уравнениям Максвелла \eqref{Gauss_E} --- \eqref{Circ_H}. Полученные выше уравнения можно получить и по-другому, не делая допущений c ротором/градиентов для \eqref{sq}: сразу подставить в уравнения \eqref{Gauss_E}, \eqref{Circ_H} наши поля из \eqref{pot} и затем воспользоваться Лоренц-калибровкой \eqref{Lorents}.

Кстати говоря, Лоренц-калибровка \eqref{Lorents} задает здесь закон сохранения заряда (в виде уравнения непрерывности). В самом деле, применим к уравнению \eqref{sq_A} дивергенцию, а к уравнению \eqref{sq_phi} дифференцирование $ \frac{1}{c}\pdd{}{t} $ и сложим эти уравнения: 

\begin{equation}\label{}
\left.
\begin{aligned}
& \square (\nabla \x \mathbf{A}) = - \dfrac{4\pi}{c} \nabla \x \mathbf{j} \\
& \square \left(  \dfrac{1}{c}\pdd{\phi}{t} \right)  = - \dfrac{4\pi}{c} \pdd{\rho}{t}
\end{aligned}
\right.
\quad \te \quad
\square \left( \nabla \x \mathbf{A}+ \dfrac{1}{c} \pdd{\phi}{t} \right)  = - \dfrac{4 \pi}{c} \left(  \nabla \x \mathbf{j} + \pdd{\rho}{t} \right) 
\end{equation}

Так как выражение под оператором Д'Аламбера равно нулю в силу калибровки \eqref{Lorents}, отсюда получаем и равенство нулю правой части, где и записан закон сохранения заряда. 



\section{Излучение диполя}

С помощью полученных результатов для потенциалов можно исследовать электромагнитное излучение диполя. 

Решение волновых уравнений \eqref{sq_A}, \eqref{sq_phi} можно записать как 

\begin{equation}\label{}
\mathbf{A}(\mathbf{r}, t) = \dfrac{1}{c} \int\limits_V \dfrac{\mathbf{j}(\mathbf{r_1}, t - \frac{|\mathbf{r - r_1}|}{c})}{|\mathbf{r} - \mathbf{r_1}|} dV_1
\end{equation}
\begin{equation}\label{}
\phi(\mathbf{r}, t) =  \int\limits_V \dfrac{\rho(\mathbf{r_1}, t - \frac{|\mathbf{r - r_1}|}{c})}{|\mathbf{r} - \mathbf{r_1}|} dV_1
\end{equation}

Здесь мы учитываем так называемые "<запаздывающие потенциалы">, т.е. берем момент времени, в который было произведено излучение, а не тот, когда волна достигла наблюдателя.

Для точечного диполя выразим ток как $ \mathbf{j} = q \mathbf{v}(t - |\mathbf{r - r_1}|/c) = q \mathbf{v}(t - r|/c) $ и будем суммировать (ведь заряды дискретны) по зарядам диполя, пренебрегая разницей расстояний от различных зарядов диполя. Тогда можно выразить $ \sum_i q_i\mathbf{v_i} = \sum_i q_i\mathbf{\dot{r_i}} = \mathbf{\dot{p}} $, и получаем для потенциала 

\begin{equation}\label{A_p}
\mathbf{A} = \dfrac{\mathbf{\dot{p}}(t - r/c)}{cr}
\end{equation}

Для  нахождения скалярного потенциала мы воспользуемся калибровкой \eqref{Lorents}, и подставив ее в векторный потенциал \eqref{A_p}, и проинтегрировав по времени обе части выражения, мы получаем

\begin{equation}\label{}
- \dfrac{1}{c} \pdd{\phi}{t} = \nabla \x \dfrac{\mathbf{\dot{p}}(t - r/c)}{cr} \te \phi = - \nabla \x \dfrac{\mathbf{p} (t - r/c)}{r}
\end{equation}

По правилу дифференцирования проведения и сложной функции, мы получаем

\begin{equation}\label{}
\phi(\mathbf{r}, t) = -\dfrac{1}{r} \nabla \x \mathbf{p} (t - r/c) - \mathbf{p} \nabla \left( \dfrac{1}{r} \right) = - \dfrac{1}{r}\mathbf{\dot{p}} \left( -\dfrac{1}{c} \nabla r \right) + \dfrac{\mathbf{pr}}{r^3} = \dfrac{\mathbf{\dot{p}r}}{cr^2} + \dfrac{\mathbf{pr}}{r^3}
\end{equation}

Мы видим, что последний член совпадает с потенциалом статического поля диполя, при этом на большом расстоянии мы можем отбросить этот член, ведь он убывает как $ \phi_{стат} \sim r^{-2} $, в то время как первый член как раз таки описывает излучение диполя в дальней зоне, убывая как $ \phi_{изл} \sim r^{-1} $, т.е. 

\begin{equation}\label{phi_p}
\phi (\mathbf{r}, t) =\dfrac{ \mathbf{r\dot{p}}(t - r/c)}{cr^2}
\end{equation}

Ну а теперь, подставив в определения полей через потенциалы \eqref{pot} полученные выражения для потенциалов колеблющегося точечного диполя, несложно получить поле излучения точеченого диполя:

\begin{equation}\label{}
\mathbf{H} = \dfrac{1}{c} \left( \dfrac{1}{r} \nabla \times \mathbf{\dot{p}}  - \mathbf{\dot{p}} \times \nabla \left( \dfrac{1}{r} \right) \right) = \dfrac{\mathbf{\ddot{p} \times r}}{c^2r^2} + \dfrac{\mathbf{\dot{p} \times r}}{cr^3}
\end{equation}
\begin{equation}\label{}
\mathbf{E} = - \nabla \left( \dfrac{ \mathbf{r\dot{p}}(t - r/c)}{cr^2} \right)  - \dfrac{1}{c} \pdd{}{t} \left( \dfrac{\mathbf{\dot{p}}}{cr} \right) = \dfrac{\mathbf{(\ddot{p}r)r} -\mathbf{\ddot{p}}r^2}{c^2r^3} 
\end{equation}

Здесь в выражении для $ \mathbf{H} $ нам нужно отбросить второй член, убывающий как $ \sim r^{-2} $, оставляя только описывающий излучение в волновой зоне член с $ \sim r^{-1} $ , и аналогично сделаем то же и для $ \mathbf{E} $ сразу. Введя вектор направления от источника к наблюдателю $ \mathbf{n} = \mathbf{r}/r $, запишем наше поле в виде

\begin{equation}\label{H, E, p}
\mathbf{H} = \dfrac{\mathbf{\ddot{p} \times n}}{c^2r}, \qquad \mathbf{E} = \dfrac{\mathbf{n \times (n \times \ddot{p})}}{c^2r}
\end{equation}

Обратим внимание на то, что электрическое и магнитное поле в волне связаны следующим соотношением: 

\begin{equation}\label{}
\mathbf{E} = \mathbf{H \times n}
\end{equation}

Т.е. вектор электрической напряжённости и магнитной ортогональны.

%%%%%%%%%%%%%%%%%%%%%%%%%%%%%%%%%%%%%%%
%	END CREDITS
%%%%%%%%%%%%%%%%%%%%%%%%%%%%%%%%%%%%%%%

\section{Заключение}

Таким образом, мы рассмотрели такие понятия, как скалярный и векторный потенциал электромагнитного поля и их использование в уравнениях Максвелла. Как пример полезности применения такого взгляда в рамках курса общей физики мы вывели излучение диполя.

Намного большее применение этот подход имеет в теоретической физике, где при помощи математического аппарата тензоров вводится понятие 4-вектора и 4-потенциала $ A = (\phi, -\mathbf{A}) $, а также тензор электромагнитного поля, выражающийся через производные по координатам и времени 4-мерного пространства. 

\section*{Список литературы}

\begin{enumerate}
	
	\item Кириченко Н.А. "<Электричество и магнетизм: учебное пособие. 2-е изд., перераб. и доп. -- М.: МФТИ, 2017. -- 378с.
	
	\item Ландау Л.Д., Лифшиц Е.М. Теоретическая физика: Учеб. пособие. В 10 т. Т. II. Теория поля. -- 7 изд., испр. -- М.: Наука. Гл. ред. физ.-мат.лит., 1988. 512 с.
	
\end{enumerate}





\end{document}